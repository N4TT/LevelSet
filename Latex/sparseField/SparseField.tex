\chapter{Sparse Field}

\section{Introduction}

The narrow band  method assumes that the computation of the SDT is so slow that it cannot be computed for every iteration. The sparse field method introduced in \cite{whitaker89} uses a fast approximation of the distance transform that makes it feasible to compute the neighborhood of the level set model for each iteration. In the sparse field method the idea of using a thin band is taken to the extreme by working on a band that is only one point wide. The band is kept track of by defining the points nearest the level set as active points. Combining them gives us the active set. (dette må muligens forandres igjen. stusser veldig på hvordan dette skal skrives)

Using only the active points to compute the derivatives would not give sufficient accuracy. Because of this, the method extends out from the active points in layers one pixel wide to create a neighborhood that is precisely the width needed to calculate the derivatives for each time step.


Several advantages to this approach are mentioned in \cite{whitaker89}. Like stated above, no more than the precise number of calculations to find the next position of the zero level set surface is used. This also results in that only those points whose values control the position of the zero level set surface are visited at each iteration, which minimizes the calculations necessary. The number of points being computed is so small that a linked-list can be used to keep track of them. \\


%This approach ensures that no unnecessary calculations are performed. 



A disadvantage of the narrow band method is that the stability at the boundaries of the band have to be maintained (by smoothing) since some points are undergoing the evolution while other neighbouring points remain fixed. The sparse field method avoid this by not letting any point entering or leaving the active set affect its value. A point enters the active set if it is adjacent to the model. As the model evolves, points that are no longer adjacent to the model are removed from the active set. This is done by defining the neighborhoods of the active set in layers and keeping the values of points entering or leaving the active set unchanged. A layer is a set of pixels represented as \(L_{i}\) where \(i\) is the city-block (manhattan) distance from the active set. The layer \(L_{0}\) represents the active set, and \(L_{\pm 1}\) reprsents pixels adjacent to the active set on both sides. Using linked lists to represents the layers and arrays (matrices) to represent distance values makes the algorithm very efficient. 

The sparse field algorithm is based on an important approximation. It assumes that points adjacent to the active points undergo the same change in value as their nearby active set neighbours. But despite this, the errors introduced by the sparse field algorithm are no worse than many other level set algorithms.\\ 

The narrow band method (og også vanlig Level Set Method) uses the same SDT for multiple iterations inside the band because reclaculating the SDT at every iteration would make the method very time-consuming. This is a tradeoff between speed and accuracy, as the accuracy of the SDT decreases with every iteration. Sparse field aproximates the SDT to be the city block distance from the active set, and recalculates this for the points in the layers at every iteration. So both methods uses tradeoffs between speed and accuracy, but the aproximations of the sparse field method has been shown to not be worse than other approaches to the level set method. (insert link)(knotete skrevet dette her).\\


Since only the grid points whose values are changing (the active points and their neighbors) are visited at each time step the growth computation time is \(d^{n-1}\), where d is the number of pixels along one dimension of the image (er dette rett?). This is the same as for parameterized models where the computation times increase with the resolution of the domain, rather than the range. 

Since we only do calculations on pixels in the active set and the neighbouring layers, the computation time increases with the size of the interface rather than the range of the domain. With comparable aproximation errors and good speed, the sparse field method is a viable approach to active shape segmentation.

%CUTOUTS FRA INTRODUCTION:

%The points (immediately?) adjacent to the level set are called active points, and all of them together are referred to as the active set. At each iteration only a thin %layer of points near the active set are visited and updated. 


\section{Overwiew of the Sparse Field method}

Like described in section \ref{sec:upwinding} (link to up-winding), the Up-Winding scheme gives the curvature in an area surrounding a point in the active set. This scheme uses both first and second order derivatives, and to calculte them it needs a 3x3x3(3D) grid of points surrounding the active point whose speed is being calculated. This creates a lower limit for the number of layers surrounding the active set. In addition to the active set which is stored in \(L{0}\) we need four lists,\(L_{1}\)  \(L_{2}\) \(L_{-1}\) \(L_{-2}\). These lists keeps track of where the points of computational significance are located at any time during execution. Like the other aproaches to the level set method, the datastructure that tracks the evolution of the interface is an array with the same dimensions as the problem domain. (kan vi skrive dette litt mer profft kanskje? for eksempel med en formel som viser at array dimension med stor R equals Image dimensions)

Its important to note that the lists are used to keep track of which points are in the active set and their neighbours, and are a redundant datastructure, separate from phi. Thus tracking the layers has no effect on the accuracy of the end result.(skriver dette fordi narrow band har et problem i interfacen, stemmer det?)


The initialization process of the interface is fairly straight forward. Like most ASM's the method starts by defining a seed point. This is usually a binary mask, of equal size as the problem domain, consisting of points defined as either inside or outside the mask. The values on the border of the mask is defined as the zero level set so the corresponding points in phi is set to 0. This set of points is the initial active set. The neighbouring layers around it is set by defining one layer at the time as the points immediately adjacent to its inner layer. Every point in each layer has its level set value (phi) set to the value of the layer it's in. Initialization is then complete. \\

Each iteration consists of four steps. First the speed of each point in the actve set is calculated and the level set is updated with the new level value of the point. Second all the layers around the active set are updated with their new position according to the change in value of its inner neighbour.  So if an active point is determined to move out of the range of the acive set, the phi value is updated, and then its neiighbouring points are all updated to be either -1 (inside) or +1 (outside) the value of the previously active point. This will make one of the points fall within the range of the active set and will update its layer to reflect this. 






